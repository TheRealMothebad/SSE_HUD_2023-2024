Doxygen is an open source tool for generating documentation directly from code! You can install it here\+: \href{https://www.doxygen.nl/download.html}{\texttt{ https\+://www.\+doxygen.\+nl/download.\+html}}

The basic idea is that you have a configuration file called \char`\"{}\+Doxyfile\char`\"{} in your code and the settings there are then used by the doxygen utility to generate HTML(like a website) documentation.

I used this tutorial to set the correct settings for our configuration file \href{https://www.doxygen.nl/download.html}{\texttt{ https\+://www.\+doxygen.\+nl/download.\+html}}

This documentation can then either be hosted on a website, or read locally. All modern browsers are able to read html files, so if you put the full file path to the docs/html/index.\+html file into your browsers search bar than it should show you the documentation page.

Whenever a change is made to the code, updating the documentation is as simple as navigating to the correct folder in the command line, and then running the \textquotesingle{}doxygen\textquotesingle{} command! 